\documentclass{beamer}

\usepackage{color}
\usepackage{graphicx}
\usepackage{minted}
\usepackage{url}

\definecolor{pms280}{HTML}{00247d}
\definecolor{pms280_2nd}{HTML}{385499}
\definecolor{pms280_4th}{HTML}{7185b6}
\definecolor{pms280_8th}{HTML}{e2e6F0}
\definecolor{pms280_compl}{HTML}{7d5800}

\usetheme{Boadilla}
\useoutertheme{infolines}
\useinnertheme{circles}
\setbeamercolor{author in head/foot}{bg=pms280,fg=white}
\setbeamercolor{title in head/foot}{bg=pms280_2nd,fg=white}
\setbeamercolor{date in head/foot}{bg=pms280_4th,fg=white}
\setbeamercolor{block title}{bg=pms280,fg=white}
\setbeamercolor{block body}{bg=pms280_8th}
\setbeamercolor{title}{fg=pms280}
\setbeamercolor{titlelike}{fg=pms280}
\setbeamercolor{itemize item}{fg=pms280}
\setbeamercolor{itemize subitem}{fg=pms280}
\setbeamertemplate{enumerate items}[default]
\setbeamercolor{enumerate item}{fg=pms280}

\author{Jonathan~K.~Vis}
\institute[LUMC]{Dept. of Human Genetics, Leiden University Medical Center}
\date{}
\title{Object Oriented Programming}

\begin{document}
\beamertemplatenavigationsymbolsempty

\begin{frame}
\titlepage
\vfill
\hfill \textcolor{pms280_compl}{\texttt{j.k.vis@lumc.nl}}
\end{frame}

\begin{frame}[fragile]{Organizing data}

\begin{itemize}
\item Lists:
\begin{minted}{python}
>>> vectors_x = [0, 1, 2]
>>> vectors_y = [0, 1, 1]
>>> (vectors_x[2], vectors_y[2])
(2, 1)
\end{minted}

\item Nested lists:
\begin{minted}{python}
>>> vectors = [[0, 0], [1, 1], [2, 1]]
>>> vectors[2]
[2, 1]
\end{minted}

\item Dictionaries:
\begin{minted}{python}
>>> vectors = [{'x': 0, 'y': 0},
               {'x': 1, 'y': 1},
               {'x': 2, 'y': 1}]
>>> vectors[2]
{'y': 1, 'x': 2}
\end{minted}
\end{itemize}
\end{frame}

\begin{frame}[fragile]{Everything is an object}
\begin{itemize}
\item Python supports many different kinds of data:
\begin{minted}{python}
42    3.14159    'Hello World!'    [1, 1, 2, 3, 5, 8]
{'name': 'Jack', 'age': 25}    True
\end{minted}

\item every \textcolor{pms280_compl}{object} has:
\begin{itemize}
\item a \textcolor{pms280_compl}{type};
\item an internal \textcolor{pms280_compl}{data representation}
(primitive or composite);
\item a set of \textcolor{pms280_compl}{procedures} (functions) for
interaction.
\end{itemize}

\item an object is an \textcolor{pms280_compl}{instance} of a type:
\begin{itemize}
\item \mintinline{python}{42} is an instance of type
\mintinline{python}{int};
\item \mintinline{python}{'Hello World!'} is an instance of type
\mintinline{python}{str}.
\end{itemize}
\end{itemize}
\end{frame}

\begin{frame}{Object oriented programming}
\begin{itemize}
\item \textcolor{pms280_compl}{create} new objects;
\item \textcolor{pms280_compl}{manipulate} objects;
\item \textcolor{pms280_compl}{destroy} objects:
\begin{itemize}
\item explicitly \mintinline{python}{del};
\item or just ``forget'' about them: Python will destroy inaccessible
objects in a process called \textcolor{pms280_compl}{garbage collection}.
\end{itemize}
\end{itemize}

Objects are \textcolor{pms280_compl}{data abstraction}:
\begin{enumerate}
\item internal data representation using
\textcolor{pms280_compl}{attributes} (member variables);
\item an interface (for interaction):
\begin{itemize}
\item procedures (member functions);
\item defines behaviour, but hides implementation.
\end{itemize}
\end{enumerate}
\end{frame}

\begin{frame}{Working with objects}
\begin{enumerate}
\item creating a \mintinline{python}{class}:
\begin{itemize}
\item define the \mintinline{python}{class} name;
\item define the \mintinline{python}{class} attributes (member
functions).
\end{itemize}

\item using the \mintinline{python}{class}:
\begin{itemize}
\item create new \textcolor{pms280_compl}{instances} of a class;
\item manipulating these instances.
\end{itemize}

\end{enumerate}
\end{frame}

\begin{frame}[fragile]{Define your own types}
Use the keyword \mintinline{python}{class} to define a new type:
\begin{itemize}
\item with as parent the \mintinline{python}{object} type;
\item and \textcolor{blue}{\mintinline{python}{Vector}} as name.
\end{itemize}

\bigskip

\begin{minted}{python}
class Vector(object):
    """2d vector class."""

    def __init__(self, x=0, y=0):
        """Initializes a new 2d vector; default: (0, 0)."""
        self.x = x
        self.y = y
\end{minted}
\end{frame}

\begin{frame}{More on attributes}
Data and procedures (functions) that ``belong'' to the class:

\begin{itemize}
\item \textcolor{pms280_compl}{Data} attributes: the objects that make up
the class:
\begin{itemize}
\item a 2d vector is made up of two numbers (\mintinline{python}{x} and
\mintinline{python}{y}).
\end{itemize}

\item \textcolor{pms280_compl}{Methods} (procedures):
\begin{itemize}
\item functions that \emph{only} work with this class;
\item how to interact with the object;
\item e.g., calculate the length of a vector.
\end{itemize}

\end{itemize}

\bigskip

\mintinline{python}{self} is the current instance of a class.

\bigskip

\mintinline{python}{def __init__(self, ...)} is a special method to
create new instances of a class.
\end{frame}

\begin{frame}[fragile]{Creating an instance of a class}
\begin{minted}{python}
>>> origin = Vector()
>>> v1 = Vector(2, 1)
>>> print(v1.x, origin.y)
\end{minted}

\bigskip

\begin{itemize}
\item don't provide anything for the \mintinline{python}{self} argument;
Python does that automatically;
\item use the \textcolor{pms280_compl}{dot} to access an attribute of an
instance;
\item \mintinline{python}{.x} is a member variable.
\end{itemize}
\end{frame}

\begin{frame}[fragile]{Hiding information --- separation of concerns}
\begin{itemize}
\item Sometimes we would like to \textcolor{pms280_compl}{hide}
attributes to the outside world, i.e., only usable inside the class.

\item in Python we prefix an attribute with \mintinline{python}{_}
(underscore) to make it \textcolor{pms280_compl}{private}:
\begin{minted}{python}
class Vector(object):
    def __init__(self, x=0, y=0):
        self.x = x
        self.y = y
        self._secret = 42    # this is allowed
\end{minted}

\item we agree not to access this attribute directly:
\begin{minted}{python}
>>> v1 = Vector(1, 2)
>>> print(v1._secret)        # this is forbidden
\end{minted}

\end{itemize}
\end{frame}

\begin{frame}[fragile]{Add a method to the \mintinline{python}{Vector} class}
\begin{minted}{python}
class Vector(object):
    def __init__(self, x=0, y=0):
        self.x = x
        self.y = y

    def distance(self, other):
        dx = (self.x - other.x) ** 2
        dy = (self.y - other.y) ** 2
        return (dx + dy) ** .5
\end{minted}

\bigskip

To use the newly created method:
\begin{minted}{python}
>>> origin.distance(v1)
\end{minted}
\end{frame}

\begin{frame}[fragile]{Print representation of an object}
\begin{minted}{python}
>>> v1 = Vector(4, 3)
>>> print(v1)
<__main__.Vector object at 0x7f41ab878450>
\end{minted}

\bigskip

\begin{itemize}
\item per default \textcolor{pms280_compl}{uninformative};
\item define the special method \mintinline{python}{def __str__(self)};
\item Python calls the \mintinline{python}{__str__} method automatically
when using the \mintinline{python}{print} function;
\item we are in control of what is printed, e.g., for the vector class:
\begin{minted}{python}
>>> print(v1)
<4, 3>
\end{minted}
\end{itemize}
\end{frame}

\begin{frame}[fragile]{Own print method}
The \mintinline{python}{__str__} function must return a string.

\begin{minted}{python}
class Vector(object):
    def __init__(self, x=0, y=0):
        self.x = x
        self.y = y

    def distance(self, other):
        dx = (self.x - other.x) ** 2
        dy = (self.y - other.y) ** 2
        return (dx + dy) ** .5

    def __str__(self):
        return '<' + str(self.x) + ', ' + str(self.y) + '>'
\end{minted}
\end{frame}

\begin{frame}[fragile]{More on types}
\begin{itemize}
\item get the type of an instance:
\begin{minted}{python}
>>> v1 = Vector(4, 3)
>>> print(type(v1))
\end{minted}
%% start minted fix
\vspace{-.08cm}
\mintinline{python}{<class}\mintinline{python}{ '__main__.Vector'>}
%% end minted fix

\item that also works for the class:
\begin{minted}{python}
>>> print(type(Vector))
<type 'type'>
\end{minted}

\item use \mintinline{python}{isinstance()} to check if an object is a
vector:
\begin{minted}{python}
>>> print(isinstance(v1, Vector))
True
\end{minted}

\item what happens here:
\begin{minted}{python}
>>> print(v1.distance(4))
\end{minted}
\end{itemize}
\end{frame}

\begin{frame}[fragile]{Special operators}
\begin{itemize}
\item define special operators like:
\mintinline{python}{+,-,==,<,>,len(),...}
see: \url{https://docs.python.org/3/reference/datamodel.html#basic-customization}

\item these can be \textcolor{pms280_compl}{overloaded} to work with your
class (keep it sensible);
\item using the double underscore notation:
\begin{minted}{python}
__add__(self, other)  # self + other
__sub__(self, other)  # self - other
__eq__(self, other)   # self == other
__lt__(self, other)   # self < other
__len__(self)         # len(self)
\end{minted}
\end{itemize}
\end{frame}

\begin{frame}[fragile]{Example of an overloaded operator}
\begin{minted}{python}
class Vector(object):
    ...

    def __add__(self, other):
        return Vector(self.x + other.x, self.y + other.y)
\end{minted}

\bigskip

\textcolor{pms280_compl}{Important:} return the right type, e.g., the
addition of two vectors results in a vector (not a string)!

\bigskip

\begin{minted}{python}
>>> v1 = Vector(1, -6)
>>> v2 = Vector(3, 4.5)
>>> print(v1 + v2)
<4, -1.5>
\end{minted}
\end{frame}

\begin{frame}{The power of OOP}
\begin{itemize}
\item \textcolor{pms280_compl}{Bundle} together objects that share:
\begin{itemize}
\item common (data) attributes;
\item methods that manipulate these attributes.
\end{itemize}

\item \textcolor{pms280_compl}{Abstract away} implementation by
specifying interfaces and behaviour;

\item Use \textcolor{pms280_compl}{inheritance} (not covered here)
to give an even nicer (layered) abstraction;

\item \textcolor{pms280_compl}{Create} own data type on top of what
Python provides;

\item \textcolor{pms280_compl}{Reuse} code: wrapping code in classes
prevents collision of function names;

\item Many \textcolor{pms280_compl}{libraries} heavily use classes.
\end{itemize}

\vfill

\begin{center}
\textcolor{pms280}{\Large \textbf{Questions?}}
\end{center}
\end{frame}

\begin{frame}[fragile]{Assignment}
\begin{minted}{python}
class Fraction(object):
    def __init__(self, numerator, denominator=1):
        self.numerator = numerator
        self.denominator = denominator
\end{minted}

\bigskip

Use the skeleton to implement a \textcolor{pms280_compl}{Fraction} class
containing two integers: \mintinline{python}{numerator} and
\mintinline{python}{denominator}.

\begin{itemize}
\item add, subtract, e.g., $\frac{1}{2} + \frac{2}{3} = \frac{7}{6}$;
\item print representation, convert to \mintinline{python}{float};
\item invert a fraction;
\item $\ldots$
\end{itemize}

\bigskip

see: \url{https://github.com/lumc-python/oop} \\
or go directly to: \textcolor{pms280_compl}{\url{https://classroom.github.com/a/8BnbL9fD}}
\end{frame}

\end{document}
