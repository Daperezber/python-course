\documentclass{beamer}

\usepackage{color}
\usepackage{graphicx}
\usepackage{minted}
\usepackage{url}

\definecolor{pms280}{HTML}{00247d}
\definecolor{pms280_2nd}{HTML}{385499}
\definecolor{pms280_4th}{HTML}{7185b6}
\definecolor{pms280_8th}{HTML}{e2e6F0}
\definecolor{pms280_compl}{HTML}{7d5800}

\usetheme{Boadilla}
\useoutertheme{infolines}
\useinnertheme{circles}
\setbeamercolor{author in head/foot}{bg=pms280,fg=white}
\setbeamercolor{title in head/foot}{bg=pms280_2nd,fg=white}
\setbeamercolor{date in head/foot}{bg=pms280_4th,fg=white}
\setbeamercolor{block title}{bg=pms280,fg=white}
\setbeamercolor{block body}{bg=pms280_8th}
\setbeamercolor{title}{fg=pms280}
\setbeamercolor{titlelike}{fg=pms280}
\setbeamercolor{itemize item}{fg=pms280}
\setbeamercolor{itemize subitem}{fg=pms280}
\setbeamertemplate{enumerate items}[default]
\setbeamercolor{enumerate item}{fg=pms280}

\author{Jonathan~K.~Vis}
\institute[LUMC]{Dept. of Human Genetics, Leiden University Medical Center}
\date{}
\title{Solutions --- Object Oriented Programming}

\begin{document}
\beamertemplatenavigationsymbolsempty

\begin{frame}
\titlepage
\vfill
\hfill \textcolor{pms280_compl}{\texttt{j.k.vis@lumc.nl}}
\end{frame}

\begin{frame}[fragile]{0. Skeleton}
\small
\texttt{fraction.py}
\begin{minted}{python}
class Fraction(object):
    def __init__(self, numerator, denominator=1):
        self._numerator = numerator
        self._denominator = denominator
\end{minted}

\vfill

\texttt{test\_fraction.py}
\begin{minted}{python}
>>> f1 = Fraction(1, 2)
\end{minted}
\end{frame}

\begin{frame}[fragile]{1. Adding a print representation}
\small
\texttt{fraction.py}
\begin{minted}{python}
class Fraction(object):
    def __init__(self, numerator, denominator=1):
        self._numerator = numerator
        self._denominator = denominator

    def __str__(self):
        return str(self._numerator) + '/' + str(self._denominator)
\end{minted}

\vfill

\texttt{test\_fraction.py}
\begin{minted}{python}
>>> print(Fraction(1, 2))
1/2
\end{minted}
\end{frame}

\begin{frame}[fragile]{2. Add the \texttt{+} operator}
\small
\texttt{fraction.py}
\begin{minted}{python}
class Fraction(object):
    def __init__(self, numerator, denominator=1):
        self._numerator = numerator
        self._denominator = denominator

    def __str__(self):
        return str(self._numerator) + '/' + str(self._denominator)

    def __add__(self, other):
        return Fraction(self._numerator * other._denominator +
                        other._numerator * self._denominator,
                        self._denominator * other._denominator)
\end{minted}

\vfill

\texttt{test\_fraction.py}
\begin{minted}{python}
>>> print(Fraction(1, 2) + Fraction(2, 3))
7/6
\end{minted}
\end{frame}

\begin{frame}[fragile]{3. Add an invert function}
\small
\texttt{fraction.py}
\begin{minted}{python}
class Fraction(object):
    ...

    def __str__(self):
        return str(self._numerator) + '/' + str(self._denominator)

    def __add__(self, other):
        return Fraction(self._numerator * other._denominator +
                        other._numerator * self._denominator,
                        self._denominator * other._denominator)

    def invert(self):
        return Fraction(self._denominator, self._numerator)
\end{minted}

\vfill

\texttt{test\_fraction.py}
\begin{minted}{python}
>>> print(Fraction(1, 2).invert())
2/1
\end{minted}
\end{frame}

\begin{frame}[fragile]{4. Conversion to a decimal representation}
\small
\texttt{fraction.py}
\begin{minted}{python}
class Fraction(object):
    ...

    def __add__(self, other):
        return Fraction(self._numerator * other._denominator +
                        other._numerator * self._denominator,
                        self._denominator * other._denominator)

    def invert(self):
        return Fraction(self._denominator, self._numerator)

    def to_float(self):
        return self._numerator / self._denominator
\end{minted}

\vfill

\texttt{test\_fraction.py}
\begin{minted}{python}
>>> print(Fraction(1, 2).to_float())
0.5
\end{minted}
\end{frame}

\begin{frame}[fragile]{5. Get the integer part of a fraction}
\small
\texttt{fraction.py}
\begin{minted}{python}
class Fraction(object):
    ...

    def invert(self):
        return Fraction(self._denominator, self._numerator)

    def to_float(self):
        return self._numerator / self._denominator

    def integer(self):
        return self._numerator // self._denominator
\end{minted}

\vfill

\texttt{test\_fraction.py}
\begin{minted}{python}
>>> print(Fraction(7, 6).integer())
1
\end{minted}
\end{frame}

\begin{frame}[fragile]{6a. Substracting}
\small
\texttt{fraction.py}
\begin{minted}{python}
class Fraction(object):
    ...

    def to_float(self):
        return self._numerator / self._denominator

    def integer(self):
        return self._numerator // self._denominator

    def __sub__(self, other):
        return Fraction(self._numerator * other._denominator -
                        other._numerator * self._denominator,
                        self._denominator * other._denominator)
\end{minted}

\vfill

\texttt{test\_fraction.py}
\begin{minted}{python}
>>> print(Fraction(1, 2) - Fraction(2, 3))
-1/6
\end{minted}
\end{frame}


\begin{frame}[fragile]{6b. Multiplication}
\small
\texttt{fraction.py}
\begin{minted}{python}
class Fraction(object):
    ...

    def integer(self):
        return self._numerator // self._denominator

    def __sub__(self, other):
        return Fraction(self._numerator * other._denominator -
                        other._numerator * self._denominator,
                        self._denominator * other._denominator)

    def __mul__(self, other):
        return Fraction(self._numerator * other._numerator,
                        self._denominator * other._denominator)
\end{minted}

\vfill

\texttt{test\_fraction.py}
\begin{minted}{python}
>>> print(Fraction(1, 2) * Fraction(2, 3))
2/6
\end{minted}
\end{frame}

\begin{frame}[fragile]{6c. Division}
\small
\texttt{fraction.py}
\begin{minted}{python}
class Fraction(object):
    ...

    def __sub__(self, other):
        return Fraction(self._numerator * other._denominator -
                        other._numerator * self._denominator,
                        self._denominator * other._denominator)

    def __mul__(self, other):
        return Fraction(self._numerator * other._numerator,
                        self._denominator * other._denominator)

    def __div__(self, other):
        return self * other.invert()
\end{minted}

\vfill

\texttt{test\_fraction.py}
\begin{minted}{python}
>>> print(Fraction(1, 2) / Fraction(2, 3))
3/4
\end{minted}
\end{frame}

\begin{frame}[fragile]{7. Simplification}
\small
\texttt{fraction.py}
\begin{minted}{python}
def gcd(a, b):
    if b == 0:
        return a
    return gcd(b, a % b)

class Fraction(object):
    ...

    def simplify(self):
        divisor = gcd(self._numerator, self._denominator)
        self._numerator = self._numerator // divisor
        self._denominator = self._denominator // divisor
        return self
\end{minted}

\vfill

\texttt{test\_fraction.py}
\begin{minted}{python}
>>> print(Fraction(2, 6).simplify())
1/3
\end{minted}
\end{frame}

\begin{frame}[fragile]{8--12. More advanced stuff}
8. It would be a good idea to \mintinline{python}{simplify()} after each
arithmetic operator. \\

\vfill

9. Here, you really need the fractions to be in a \emph{normal} form. \\

\vfill
10. Probably (for printing reasons) you would like for the numerator to
be negative and not the denominator, avoid \mintinline{python}{1/-4}. And
convert to positive when both the numerator and denominator are negative. \\

\vfill

11. Something like:
{
\small
\begin{minted}{python}
def __mul__(self, other):
    if isinstance(other, int):
        return Fraction(self._numerator * other, self._denominator)
    return Fraction(self._numerator * other._numerator,
                    self._denominator * other._denominator)
\end{minted}
}
\vfill

12. Using \mintinline{python}{isinstance(numerator, int)} etc. in the
\mintinline{python}{__init__} function.
\end{frame}

\begin{frame}[fragile]{General remarks}
\begin{itemize}
\item As always, more docstrings and \emph{more} comments;
\item All \mintinline{python}{import}s on the top of the file;
\item \mintinline{python}{a is b} does \emph{not} check for numerical
equality: use \mintinline{python}{==};
\item \emph{Never} use bitwise operators: \mintinline{python}{&, |, ^, ~}:
use \mintinline{python}{and or not};
\item Prefer not calling the special functions, e.g.,
\mintinline{python}{__add__} directly: use \mintinline{python}{+};
\item Try to make your code look \emph{beautiful}.
\end{itemize}
\end{frame}
\end{document}
